% Options for packages loaded elsewhere
\PassOptionsToPackage{unicode}{hyperref}
\PassOptionsToPackage{hyphens}{url}
%
\documentclass[
]{article}
\usepackage{lmodern}
\usepackage{amssymb,amsmath}
\usepackage{ifxetex,ifluatex}
\ifnum 0\ifxetex 1\fi\ifluatex 1\fi=0 % if pdftex
  \usepackage[T1]{fontenc}
  \usepackage[utf8]{inputenc}
  \usepackage{textcomp} % provide euro and other symbols
\else % if luatex or xetex
  \usepackage{unicode-math}
  \defaultfontfeatures{Scale=MatchLowercase}
  \defaultfontfeatures[\rmfamily]{Ligatures=TeX,Scale=1}
\fi
% Use upquote if available, for straight quotes in verbatim environments
\IfFileExists{upquote.sty}{\usepackage{upquote}}{}
\IfFileExists{microtype.sty}{% use microtype if available
  \usepackage[]{microtype}
  \UseMicrotypeSet[protrusion]{basicmath} % disable protrusion for tt fonts
}{}
\makeatletter
\@ifundefined{KOMAClassName}{% if non-KOMA class
  \IfFileExists{parskip.sty}{%
    \usepackage{parskip}
  }{% else
    \setlength{\parindent}{0pt}
    \setlength{\parskip}{6pt plus 2pt minus 1pt}}
}{% if KOMA class
  \KOMAoptions{parskip=half}}
\makeatother
\usepackage{xcolor}
\IfFileExists{xurl.sty}{\usepackage{xurl}}{} % add URL line breaks if available
\IfFileExists{bookmark.sty}{\usepackage{bookmark}}{\usepackage{hyperref}}
\hypersetup{
  pdftitle={Programming with R},
  pdfauthor={Claire Vandiedonck \& Jacques van Helden},
  hidelinks,
  pdfcreator={LaTeX via pandoc}}
\urlstyle{same} % disable monospaced font for URLs
\usepackage[margin=1in]{geometry}
\usepackage{graphicx}
\makeatletter
\def\maxwidth{\ifdim\Gin@nat@width>\linewidth\linewidth\else\Gin@nat@width\fi}
\def\maxheight{\ifdim\Gin@nat@height>\textheight\textheight\else\Gin@nat@height\fi}
\makeatother
% Scale images if necessary, so that they will not overflow the page
% margins by default, and it is still possible to overwrite the defaults
% using explicit options in \includegraphics[width, height, ...]{}
\setkeys{Gin}{width=\maxwidth,height=\maxheight,keepaspectratio}
% Set default figure placement to htbp
\makeatletter
\def\fps@figure{htbp}
\makeatother
\setlength{\emergencystretch}{3em} % prevent overfull lines
\providecommand{\tightlist}{%
  \setlength{\itemsep}{0pt}\setlength{\parskip}{0pt}}
\setcounter{secnumdepth}{-\maxdimen} % remove section numbering

\title{Programming with R}
\usepackage{etoolbox}
\makeatletter
\providecommand{\subtitle}[1]{% add subtitle to \maketitle
  \apptocmd{\@title}{\par {\large #1 \par}}{}{}
}
\makeatother
\subtitle{DUBii -- Statistics with R}
\author{Claire Vandiedonck \& Jacques van Helden}
\date{2020-03-04}

\begin{document}
\maketitle

{
\setcounter{tocdepth}{3}
\tableofcontents
}
\hypertarget{exercice-1-test}{%
\subsection{Exercice 1~: test}\label{exercice-1-test}}

\begin{itemize}
\item
  Créez un vecteur de 100 valeurs tirées aléatoirement selon une loi
  normale de moyenne 4 et d'écart type 5.
\item
  Identifiez les indices des valeurs strictement supérieures à 3 et
  récupérez les valeurs correspondantes.
\item
  Testez si la somme de ces valeurs est supérieure à 40, à 30 ou à 20 et
  affichez un message adéquat selon chaque éventualité.
\end{itemize}

\textbf{Fonctions à utiliser :} rnorm(), which(),sum(), if(),else
if(),else(), cat() ou print()

\hypertarget{exercice-2-cruxe9ation-dune-boucle-simple}{%
\subsection{Exercice 2~: création d'une boucle
simple}\label{exercice-2-cruxe9ation-dune-boucle-simple}}

\begin{itemize}
\tightlist
\item
  Créez une boucle de 10 itérations \texttt{i} qui affiche à chaque
  itération l'indice \texttt{i}.
\item
  Calculez au fur et à mesure la somme cumulée des indices dans un
  vecteur.
\item
  Affichez la somme cumulée finale.
\end{itemize}

\textbf{Fonctions à utiliser~:} for(), cat() ou print()

\hypertarget{exercice-3-cruxe9ation-dune-fonction}{%
\subsection{Exercice 3~: création d'une
fonction}\label{exercice-3-cruxe9ation-dune-fonction}}

\begin{itemize}
\tightlist
\item
  Créez une fonction \texttt{calculSomme} qui calcule la somme de deux
  variables x et y passées en argument.
\item
  Testez la fonction.
\end{itemize}

\textbf{Fonctions à utiliser~:} function(), return()

\hypertarget{exercice-4-cruxe9ation-dune-fonction-avec-des-tests-utilisation-de-cette-fonction-de-maniuxe8re-ituxe9rative}{%
\subsection{Exercice 4: création d'une fonction avec des tests,
utilisation de cette fonction de manière
itérative}\label{exercice-4-cruxe9ation-dune-fonction-avec-des-tests-utilisation-de-cette-fonction-de-maniuxe8re-ituxe9rative}}

\begin{itemize}
\tightlist
\item
  Ecrivez une fonction \texttt{calculTarif} qui prend pour argument un
  âge et affiche ``demi-tarif'' si l'âge est inférieur à 12 ans, ``tarif
  sénior'' si l'âge est supérieur ou égal à 60 ans et ``plein tarif''
  sinon.
\item
  Testez votre fonction pour des personnes de 5, 65, 85, 41, 23 et 47
  ans.
\end{itemize}

\textbf{Fonctions à utiliser~:}

\begin{itemize}
\tightlist
\item
  \texttt{function()},
\item
  \texttt{return()},
\item
  \texttt{print()},
\item
  \texttt{c()}
\end{itemize}

\textbf{Instructions conditionnelles: } - \texttt{if} - \texttt{else} -
\texttt{else\ if} - \texttt{for}

\hypertarget{exercice-5-cruxe9ation-dune-fonction-avec-compteur-de-boucle}{%
\subsection{Exercice 5: création d'une fonction avec compteur de
boucle}\label{exercice-5-cruxe9ation-dune-fonction-avec-compteur-de-boucle}}

\begin{itemize}
\item
  Ecrivez une fonction \texttt{sumCumul} qui calcule la somme cumulée
  des nombres entiers compris entre deux bornes \(a\) et \(b\) que vous
  mettrez en arguments \texttt{start} et \texttt{end}.
\item
  De plus, toutes les 10 boucles, vous affichez la valeur de l'entier
  ajouté, sinon vous affichez un point \texttt{.}.
\item
  Testez la fonction avec les entiers entre 3 et 55 par exemple.
\item
  Améliorez la fonction en ajoutant un paramètre \texttt{interval}
  correspondant à l'incrément de boucles entre les affichages des
  entiers (dans la fonction précédente, cet incrément était de 10)
  Testez la fonction avec les entiers entre 3 et 55 et un intervalle de
  15 par exemple.
\end{itemize}

\textbf{Fonctions à utiliser~:}

\begin{itemize}
\tightlist
\item
  \texttt{function()},
\item
  \texttt{return()},
\item
  \texttt{cat()},
\item
  \texttt{seq()}
\end{itemize}

\end{document}
